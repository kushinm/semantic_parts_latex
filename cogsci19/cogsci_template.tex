% 
% Annual Cognitive Science Conference
% Sample LaTeX Paper -- Proceedings Format
% 

% Original : Ashwin Ram (ashwin@cc.gatech.edu)       04/01/1994
% Modified : Johanna Moore (jmoore@cs.pitt.edu)      03/17/1995
% Modified : David Noelle (noelle@ucsd.edu)          03/15/1996
% Modified : Pat Langley (langley@cs.stanford.edu)   01/26/1997
% Latex2e corrections by Ramin Charles Nakisa        01/28/1997 
% Modified : Tina Eliassi-Rad (eliassi@cs.wisc.edu)  01/31/1998
% Modified : Trisha Yannuzzi (trisha@ircs.upenn.edu) 12/28/1999 (in process)
% Modified : Mary Ellen Foster (M.E.Foster@ed.ac.uk) 12/11/2000
% Modified : Ken Forbus                              01/23/2004
% Modified : Eli M. Silk (esilk@pitt.edu)            05/24/2005
% Modified : Niels Taatgen (taatgen@cmu.edu)         10/24/2006
% Modified : David Noelle (dnoelle@ucmerced.edu)     11/19/2014

%% Change "letterpaper" in the following line to "a4paper" if you must.

\documentclass[10pt,letterpaper]{article}

\usepackage{cogsci}
\usepackage{pslatex}
\usepackage{apacite}


\title{semantic parts}
 
\author{{\large \bf Temp} \AND {\large \bf Temp}
  }







\begin{document}

\maketitle 


\begin{abstract}
This is where our abstract will go

\textbf{Keywords:} 
sketching; cognitive science; perception
\end{abstract}



\section{Introduction}
This is where our introduction will go




\section{Methods}


\subsection{Stimuli}

 From \_\_ we obtained 1198 sketches of 32 real-world objects belonging to 4 basic-level categories : cars, chairs, dogs, and birds. There were 8 diverse exemplars per category.
The sketches were made during a 2-player ‘Pictionary’-style game, where a sketcher had to create sketches of target images so as to distinguish the target from 3 distractor images. The viewer had to guess which of the 4 images the sketch represented.
There were 2 main context conditions in the original experiment - close and far. In the close condition the target image and the distractors belonged to the same basic-level category. In the far condition, the target and each of the distractors belonged to a different basic-level category.

The sketches were represented as scalable vector graphics (SVG) images. We were interested in having participants label strokes in a given sketch. We defined a stroke to be equivalent to a single cubic Bézier curve, i.e., a Bézier curve with two fixed end points and two control points to control curvature.



\subsection{Participants}

We recruited a total of 326 participants via Amazon Mechanical Turk (AMT). Participants were paid a base amount of \$1 and were given an additional bonus of \$0.002 for every stroke they annotated. In addition to this, they were given a \$0.02 bonus for every sketch for which they labeled all strokes. 


\subsection{Annotation Procedure}

To collect fine-grained annotations of our sketches, we implemented a web-based Javascript annotation tool. 
Our task consisted of one demonstration trial and 10 annotation trials. We provided participants with a sketch to be annotated on a canvas as well as a category-specific menu of labels, which they were encouraged to use for the annotation task. We also provided them with the option of entering their own labels through a free-response box, which could be accessed by clicking an option called ‘Other’ on the menu. 
The participant could see the constituent strokes of the sketch by hovering their cursor over different parts of the canvas.  To label a stroke, they could click on it or click and drag their cursor over several strokes. This would highlight the selected strokes. Next, they would choose a label from the menu or submit their own through the free response box. The highlighted strokes would change color to match the background color of the selected part-word in the menu. This was to provide a visual aid to the participant as to what strokes had already been labeled and as what part. Participants were encouraged to conduct their labeling of strokes in bouts — they were to highlight all the strokes corresponding to a single instance of a part before selecting a label from the menu. A running counter next to each part would indicate how many bouts of annotation had been completed for that part. A running progress meter below the canvas would inform the sketcher as to how many strokes they had labeled out of the total number of strokes in the sketch. Once they were done with sketch, participants could go on to the next trial by clicking a ‘Next’ button under the canvas. They could choose to continue to the next trial without labeling every stroke in a sketch, but they would lose out on the completion bonus as well as the amount they would have earned for labeling the remaining strokes.
After receiving instructions on how to use the tool, participants were told to test out the interface in a demonstration trial. For this trial alone they could not proceed unless they labeled every stroke in the sketch. The sketch shown for the demo trial was the same for every participant.

\noindent In total we collected 3608 annotations


\subsection{Analysis}

Temp

\section{Results}


\section{Discussion}

\section{Formalities, Footnotes, and Floats}

Use standard APA citation format. Citations within the text should
include the author's last name and year. If the authors' names are
included in the sentence, place only the year in parentheses, as in
\citeA{NewellSimon1972a}, but otherwise place the entire reference in
parentheses with the authors and year separated by a comma
\cite{NewellSimon1972a}. List multiple references alphabetically and
separate them by semicolons
\cite{ChalnickBillman1988a,NewellSimon1972a}. Use the
``et~al.'' construction only after listing all the authors to a
publication in an earlier reference and for citations with four or
more authors.


\subsection{Footnotes}

Indicate footnotes with a number\footnote{Sample of the first
footnote.} in the text. Place the footnotes in 9~point type at the
bottom of the column on which they appear. Precede the footnote block
with a horizontal rule.\footnote{Sample of the second footnote.}


\subsection{Tables}

Number tables consecutively. Place the table number and title (in
10~point) above the table with one line space above the caption and
one line space below it, as in Table~\ref{sample-table}. You may float
tables to the top or bottom of a column, or set wide tables across
both columns.

\begin{table}[!ht]
\begin{center} 
\caption{Sample table title.} 
\label{sample-table} 
\vskip 0.12in
\begin{tabular}{ll} 
\hline
Error type    &  Example \\
\hline
Take smaller        &   63 - 44 = 21 \\
Always borrow~~~~   &   96 - 42 = 34 \\
0 - N = N           &   70 - 47 = 37 \\
0 - N = 0           &   70 - 47 = 30 \\
\hline
\end{tabular} 
\end{center} 
\end{table}


\subsection{Figures}

All artwork must be very dark for purposes of reproduction and should
not be hand drawn. Number figures sequentially, placing the figure
number and caption, in 10~point, after the figure with one line space
above the caption and one line space below it, as in
Figure~\ref{sample-figure}. If necessary, leave extra white space at
the bottom of the page to avoid splitting the figure and figure
caption. You may float figures to the top or bottom of a column, or
set wide figures across both columns.

\begin{figure}[ht]
\begin{center}
\fbox{CoGNiTiVe ScIeNcE}
\end{center}
\caption{This is a figure.} 
\label{sample-figure}
\end{figure}


\section{Acknowledgments}

Place acknowledgments (including funding information) in a section at
the end of the paper.


\section{References Instructions}

Follow the APA Publication Manual for citation format, both within the
text and in the reference list, with the following exceptions: (a) do
not cite the page numbers of any book, including chapters in edited
volumes; (b) use the same format for unpublished references as for
published ones. Alphabetize references by the surnames of the authors,
with single author entries preceding multiple author entries. Order
references by the same authors by the year of publication, with the
earliest first.

Use a first level section heading, ``{\bf References}'', as shown
below. Use a hanging indent style, with the first line of the
reference flush against the left margin and subsequent lines indented
by 1/8~inch. Below are example references for a conference paper, book
chapter, journal article, dissertation, book, technical report, and
edited volume, respectively.

\nocite{ChalnickBillman1988a}
\nocite{Feigenbaum1963a}
\nocite{Hill1983a}
\nocite{OhlssonLangley1985a}
% \nocite{Lewis1978a}
\nocite{Matlock2001}
\nocite{NewellSimon1972a}
\nocite{ShragerLangley1990a}


\bibliographystyle{apacite}

\setlength{\bibleftmargin}{.125in}
\setlength{\bibindent}{-\bibleftmargin}

\bibliography{CogSci_Template}


\end{document}
